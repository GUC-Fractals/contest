
\documentclass[a4paper]{article}
\usepackage[english]{babel}
\usepackage{graphicx}
\usepackage{multicol}
\usepackage{amsmath}
\usepackage{hyperref}
\usepackage{amsthm}
\usepackage{geometry}
\geometry{a4paper}
\usepackage{fancyhdr}
\usepackage{xcolor}
\usepackage{amssymb}
\usepackage{multicol}
\theoremstyle{definition}
\newtheorem{exmp}{Example}[section]
\newtheorem{theorem}{Theorem}
\usepackage{tcolorbox}

\begin{document}
\author{\textbf{Fractals}}
\title{\textbf{Contest}}
\maketitle
\noindent


\begin{enumerate}
    \item
Given two integers, define an operation $*$ such that if $a$ and $b$ are integers, then $a*b$ is an integer.
The operation $*$ has the following properties:
\begin{itemize}
    \item $a*a = 0$ for all integers $a$;
    \item $(ka+b)*a = b* a$ for all integers $a$, $b$, and $k$;
    \item $0 \leq b*a \le a$.
    \item $0 \leq b \le a$ then $b*a = b$.
\end{itemize}
Find $2021 * 16$

\begin{tcolorbox}[width=\linewidth, sharp corners=all, colback=white!95!black]
$\textit{ Sol. }$
$ 2021 = 16(126) + 5 $ so $2021 * 16 = (16(126) + 5) * 16 = 5 * 16
= \boxed{5}$

\end{tcolorbox}


\item It’s currently $6:00$ on a $12$ hour clock. What time will be shown on the clock $100$ hours from
now? Express your answer in the form $hh : mm$.

\begin{tcolorbox}[width=\linewidth, sharp corners=all, colback=white!95!black]

$\textit{ Sol. }$
We note that adding any multiple of 12 hours does not change the time on the clock.
We see that 100 divided by 12 has remainder 4. So that means the time 100 hours from now is
equal to the time 4 hours from now, which is \boxed{$10:00$}.
\end{tcolorbox}

\item We call a positive integer \textit{binary-okay} if at least half of the digits in its binary (base 2) representation are 1’s, but no two 1s are consecutive. For example, $10_{10} = 1010_2$ and $5_{10} = 101_2$
are both binary-okay, but $16_{10} = {10000}_2$ and $11_{10} = {1011}_2$ are not. Compute the number of
binary-okay positive integers less than or equal to $2020$ (in base $10$).

\begin{tcolorbox}[width=\linewidth, sharp corners=all, colback=white!95!black]

$\textit{ Sol. }$
Begin by noting that ${2020}_{10}$ has 11 digits in binary, as $2^{10} < 2020 < 2^{11} $. There are
two cases: either the positive integer has an odd number of digits, or the positive integer has an
even number of digits.

\textbf{Case 1}: Odd number of digits:
If a positive integer has an odd number of binary digits, it must begin with a 1 and alternate
between 0's and 1's, since it will necessarily have one more 1 than 0s. (If it were to have any
more 1s, then by Pigeonhole Principle, at least two of them must be adjacent.)


\textbf{Case 2}: Even number of digits:
If it has an even number of binary digits, then exactly half of them must be 1s by a similar
argument. Precisely, if it has 2n digits, then there are n ways to arrange the 1s with no two
consecutive. (If there are n 1s, there must be n - 1 0s inserted between the 1s to prevent the
1s from being adjacent.) The final 0 can go anywhere except at the beginning, so there are n
positions for the final 0.
Thus, there are 1 + 1 + 1 + 1 + 1 + 1 = 6 binary-okay numbers with an odd number of digits (as
$101010101012 < 1024 + 512 = 1536 < 2020)$ and 5 + 4 + 3 + 2 + 1 = 15 binary-okay numbers
with an even number of digits. Altogether, we have \boxed{21} binary-okay numbers under 2020.
\end{tcolorbox}

\item Let $n$ be an integer such that $n^4 -2n^3 -n^2 +2n + 2$ is a prime number. What's the sum
of all possible $n$?


\begin{tcolorbox}[width=\linewidth, sharp corners=all, colback=white!95!black]

$\textit{ Sol. }$
We can see that this expression must be
an even number since $n^2$ and $n^4$ must be the same parity.
since The only even prime is 2, so the sum is \boxed{2}(by Viete’s theorem)
\end{tcolorbox}
\item Find the number of subsets $S$ of $\{1,2,...,10\}$ such that no two of the elements in $S$ are consecutive

\begin{tcolorbox}[width=\linewidth, sharp corners=all, colback=white!95!black]

$\textit{ Sol. }$
The subsets can be interpted as $n-words$ from
The alphapet $ \{0,1\}'$.
Let $a_n$ be the number of words with no consecutive ones.
Then a word can start from 0 and proceed with in $a_{n-1}$
ways or start with $10$ and proceed with in $a_{n-2}$ ways.
Therefore $a_n = a_{n-1} + a_{n-2}$ So $a_10 =$ \boxed{144}
\end{tcolorbox}
\item Can an $8\times8$ board be coverd by $15$ $1\times4$ rectangles and only one $2\times2$ square without overlaping? prove your answer.


\begin{tcolorbox}[width=\linewidth, sharp corners=all, colback=white!95!black]

$\textit{ Sol. }$
No.
\end{tcolorbox}

% \newpage
\item A $3  \times 3$ magic square is a grid of distinct numbers whose rows, columns, and diagonals
all add to the same integer sum. Connie creates a magic square whose sum is $N$, but her
keyboard is broken so that when she types a number, one of the digits ($0 - 9$) always appears
as a different digit (e.g. if the digit $8$ always appears as $5$, the number $18$ will appear as $15$).
The altered square is shown below. Find $N$.

\begin{table}[h]
    \begin{center}

\begin{tabular}{|l|l|l|}
\hline
9  & 11 & 10 \\ \hline
18 & 17 & 6  \\ \hline
14 & 11 & 15 \\ \hline
\end{tabular}
\end{center}
\end{table}

\begin{tcolorbox}[width=\linewidth, sharp corners=all, colback=white!95!black]

$\textit{ Sol. }$
First, notice that 11 appears in the square twice. This is not possible in the
original magic square, because the numbers must be distinct, so we conclude that the output
of the broken key is 1.
Next, we see that the digits 0, 1, 4, 5, 6, 7, 8, 9 are all present in the square. Therefore, the
broken key must be 2 or 3. We then calculate the sum of the rows and columns as shown:
\end{tcolorbox}
\begin{table}[h]
    \begin{center}

\begin{tabular}{|l|l|l|l}
\hline
9  & 11 & 10  & \textbf{30}\\ \hline
18 & 17 & 6  & \textbf{41}\\ \hline
14 & 11 & 15 & \textbf{40}\\ \hline
\textbf{41} & \textbf{39} & \textbf{31}
\end{tabular}
\end{center}
\end{table}


\begin{tcolorbox}[width=\linewidth, sharp corners=all, colback=white!95!black]

each row has a distinct sum, so there must be at least two altered squares. the middle row
is 18 + 17 + 6 = 41. since none of the ones digits are 1, the ones digit of n is the same as
the ones digit of 41.
the ones digits in the 9 + 11 + 10 = 30 row needs to sum up to end in 1, so the broken key
must be 2. similarly, the ones digit in 14 + 11 + 15 = 40 needs to be 1, so both of the 11’s
actually end with a 2
\end{tcolorbox}

\begin{table}[h]
    \begin{center}

\begin{tabular}{|l|l|l|}
\hline
9  & \textbf{12} & 10 \\ \hline
18 & 17 & 6  \\ \hline
14 & \textbf{12} & 15 \\ \hline
\end{tabular}
\end{center}
\end{table}

\begin{tcolorbox}[width=\linewidth, sharp corners=all, colback=white!95!black]

They cannot both be 12’s, so one of them is a 22. This means that the sum of the middle
column is either 12 + 22 + 17 = 51 or 12 + 22 + 27 = 61. Looking at the top row, even if all
of the 1’s are changed into 2’s, the maximum value of the row is 9 + 22 + 20 = 51. Therefore,
the sum of each row and column is \boxed{$N$ = 51} .
The original square is shown below, with the modified digits in bold:
\end{tcolorbox}
\begin{table}[h]
    \begin{center}

\begin{tabular}{|l|l|l|}
\hline
9  & \textbf{22} & \textbf{2}0 \\ \hline
\textbf{2}8 & 17 & 6  \\ \hline
14 & 1\textbf{2} & \textbf{2}5 \\ \hline
\end{tabular}
\end{center}
\end{table}
\newpage
\item Compute
$$\sum_{k \ge 0} {1000 \choose 3k}$$
\begin{tcolorbox}[width=\linewidth, sharp corners=all, colback=white!95!black]
    $\textit{ Sol. }$
    We can rewrite the sum as
    \[
        \sum_{k \ge 0} {1000 \choose n} f(n)
    \]
    where
    \[
        f(n) = \begin{cases}
            1 & n \equiv 0 \ \  (mod  \ 3) \\
            0 & otherwise
        \end{cases}
    \]
    now we can have
     \[
        f(n) = \dfrac{1}{3} (1^n + \omega^n + \omega ^{2n})
    \]
    where $\omega = e^{\frac{2}{3}\pi i}$ a cubic root of unity.
    sastifying the equation $\omega ^ 2 + \omega + 1 = 0$.
    Thus we have
    \[
        \sum_{n \ge 0} {1000 \choose n} f(n) = \dfrac{1}{3} \sum_{n \ge 0} {1000
        \choose n} (1^n + \omega^n + \omega ^{2n}) =
    \]
    \[
        \dfrac{1}{3} \sum_{n \ge 0} {1000 \choose n} + \dfrac{1}{3} \sum_{n \ge 0} {1000 \choose n} \omega^n + \dfrac{1}{3} \sum_{n \ge 0} {1000 \choose n} \omega ^{2n} =
    \]
    \[
        \sum_{n \ge 0} {1000 \choose n} f(n) = \dfrac{1}{3}[(1+1)^{1000} + (1+\omega)^{1000} + (1+\omega^2)^{1000}]
        = \dfrac{1}{3}(2^{1000} - 1)
    \]
\end{tcolorbox}
\end{enumerate}

\end{document}
